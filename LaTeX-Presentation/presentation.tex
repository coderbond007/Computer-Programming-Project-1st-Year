% Beamer Presentation
\documentclass{beamer}

\mode<presentation> {

\usetheme{Madrid}
}

\usepackage{graphicx} % Allows including images
\usepackage{booktabs} % Allows the use of \toprule, \midrule and \bottomrule in tables

\title[Project]{ITWS Project Presentation} % The short title appears at the bottom of every slide, the full title is only on the title page

\author{Pradyumn Agrawal , Susmit Lavania , Akash Bhudaulia} % Your name
\institute % Your institution as it will appear on the bottom of every slide, may be shorthand to save space
{
The LNM Institute of Information Technology , Jaipur\\
\medskip
\textit{15UCS096 , 15UCS148 , 15UEC009}
}
\date{\today} % Date, can be changed to a custom date

\begin{document}

\begin{frame}
\titlepage % Print the title page as the first slide
\end{frame}

\section{First Section}
\subsection{Subsection Example}

\begin{frame}
\frametitle{Sorting Comparisons Count}
In this project,we have made the code from the given algorithms and created our own code from the given theory about the code of selection sort. Through this project we got to know about the various sorting algorithms and how to code them in C language. Sorting algorithms include \textbf{Bubble Sort}, \textbf{Insertion Sort}, \textbf{Merge Sort}, \textbf{Quick Sort} and \textbf{Selection Sort}.\\~\\

We plotted the no. of comparisons vs total data no. of input data \textbf{(Count vs n)} of each sorting algorithm and got to know about the variations of time required for same data with different sorting algorithms.
\end{frame}

%------------------------------------------------


\begin{frame}
\frametitle{Important Points}
\begin{itemize}
\item Got to know about the sorting algorithms of various types and number of comparisons required by each algorithm.
\item Plotting of graph by the data of number of comaparisons required by the algorithm for n number of data.
\item Plotting graph using \textbf{GNU Plot} and exporting data in a file of \textbf{.txt} format using bash script.
\item Getting a rough idea about the efficiency of each algorithm.
\item \textbf{Result Achieved:} Merge Sort is the \textbf{best algorithm} in all the algorithms we worked out so far with less time complexity.
\end{itemize}
\end{frame}

\begin{frame}
\Huge{\centerline{The End}}
\end{frame}

%----------------------------------------------------------------------------------------

\end{document}
